%---------------------------------------------------------------------------
%
% :File:       hwtemplate.tex
% :Purpose:    Homework for Model of software systems
%
%---------------------------------------------------------------------------

\documentclass[11pt]{article}
\usepackage{fullpage}
\usepackage{amsmath,amsfonts,enumerate}
\usepackage{zed-csp}

% Define admin constants
\def\name{}
\def\course{Models of Software Systems}
\def\due{19.09.2018}
\def\title{Assignment 1}

\begin{document}
%----------------------------
% BEGIN HEADER PORTION
%----------------------------
\hspace*{-12mm} 
\framebox{
 \begin{minipage}[t]{\textwidth}
  \vspace{1ex}
	{\large\bf  \course} \hfill {\large\bf Name: \name}\\[1ex]
	{\bf \title \hfill   Due: \due}
  \vspace{1ex}
 \end{minipage}
}
\vskip 0.1in
%----------------------------
% END HEADER PORTION
%----------------------------

\section{Lambda Language}
Consider a language with alphabet $\{ \lambda, \bullet, (, ), x, y, z \}$ and syntax

 \begin{syntax}
     expression  & =  & variable~name | expression, expression\\
     & | & $``$ \lambda $''$, variable~name, $``$ \bullet $''$, expression\\
     & | & ``($"$, expression, ``)$"$;
     \also
     variable~name & = & $``$x$''$ | $``$y$''$ | $``$z$''$;
 \end{syntax}

Are the following wffs of the language? For those that are not briefly explain why.

\begin{enumerate}[(a)]
 \item $\lambda x \bullet yz$
Yes, wff
 \item $\lambda \bullet x \lambda \bullet y$ 
No, it is not, as $ \lambda $ should be followed by a variable~name and not a $ \bullet $
 \item $\lambda y \bullet x \bullet z$ 
Yes, wff
 \item $\lambda x \bullet x(yz)$
Yes, wff
 \item $\lambda x \bullet \lambda y \bullet xyz$
Yes, wff
\end{enumerate}

\section{Stars, Derivation}
Using the {\em Stars} formal system that was discussed during the Lab 1 (see slide 11) formally show that
 ${\star}{\diamond}{\star}{\star}{\circ}{\star}{\star}{\star}{\star}{\star}\ {\vdash}\ {\star}{\diamond}{\star}{\star}{\star}{\star}{\circ}{\star}{\star}{\star}{\star}{\star}{\star}{\star}$

\section{Stars, Incompleteness}
During the Lab 1 the {\em Stars} formal system (see slides 11-12) was interpreted as a system for adding certain positive integers. For example, $1 + 3 = 4$ could be proved a theorem of {\em Stars}.

\begin{enumerate}[(a)]
 \item Augment {\em Stars} so that you can prove statements such as $3+4=7$ and $15 + 2 = 17$ to be theorems.
 You need to handle only expressions involving the addition of positive integers.
 \textsc{Note:} Your answer should include the alphabet, syntax, inference system, and interpretation.

 \item Prove that $3 + 4 = 7$ is a theorem of the augmented system.
\end{enumerate}

\section{Propositional logic}

\begin{enumerate}[1.]
	\item Construct a truth table for each of the following:
	
	\begin{enumerate}[a.]
		\item (Example) $p \land (p \lor q)$
		\begin{displaymath}
		\begin{array}{cccc} \hline
		p & q & (p \lor q) & \quad p \land (p \lor q)\\
		\hline
		T & T & T & T\\
		T & F & T & T\\
		F & T & T & F\\
		F & F & F & F\\
		\hline \end{array}
		\end{displaymath}
		
		\item $(p \land q) \lor (\neg p \land \neg q)$
		
		\item $p \implies (q \land r)$
		
		\item $(p \implies (q \land r)) \Leftrightarrow ((p \implies q) \land (p \implies r))$
		
	\end{enumerate}
	\item Which of the above sentences are:
	\begin{enumerate}
		\item tautologies?
		\item satisfiable?
	\end{enumerate}
	\item For every satisfiable sentence detected in 2., provide z3 code that proves the satisfiability.\\
	(Example) For ``1.a.'':
	\begin{verbatim}
	(declare-const p Bool)
	(declare-const q Bool)
	(define-fun conjecture () Bool
	    (and p (or p q)))
	(assert conjecture)
	(check-sat)
	\end{verbatim}
	\item For every tautology detected in 2., provide z3 code that proves the tautology.
\end{enumerate}

\end{document}
